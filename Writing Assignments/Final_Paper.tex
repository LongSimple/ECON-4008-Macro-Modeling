\documentclass[]{article}
\usepackage{lmodern}
\usepackage{amssymb,amsmath}
\usepackage{ifxetex,ifluatex}
\usepackage{fixltx2e} % provides \textsubscript
\ifnum 0\ifxetex 1\fi\ifluatex 1\fi=0 % if pdftex
  \usepackage[T1]{fontenc}
  \usepackage[utf8]{inputenc}
\else % if luatex or xelatex
  \ifxetex
    \usepackage{mathspec}
  \else
    \usepackage{fontspec}
  \fi
  \defaultfontfeatures{Ligatures=TeX,Scale=MatchLowercase}
\fi
% use upquote if available, for straight quotes in verbatim environments
\IfFileExists{upquote.sty}{\usepackage{upquote}}{}
% use microtype if available
\IfFileExists{microtype.sty}{%
\usepackage{microtype}
\UseMicrotypeSet[protrusion]{basicmath} % disable protrusion for tt fonts
}{}
\usepackage[margin=1in]{geometry}
\usepackage{hyperref}
\hypersetup{unicode=true,
            pdftitle={An Analysis of the Premium Associated with Bitcoin},
            pdfkeywords={Cryptocurrency, Equity Premium, Finance},
            pdfborder={0 0 0},
            breaklinks=true}
\urlstyle{same}  % don't use monospace font for urls
\usepackage{graphicx,grffile}
\makeatletter
\def\maxwidth{\ifdim\Gin@nat@width>\linewidth\linewidth\else\Gin@nat@width\fi}
\def\maxheight{\ifdim\Gin@nat@height>\textheight\textheight\else\Gin@nat@height\fi}
\makeatother
% Scale images if necessary, so that they will not overflow the page
% margins by default, and it is still possible to overwrite the defaults
% using explicit options in \includegraphics[width, height, ...]{}
\setkeys{Gin}{width=\maxwidth,height=\maxheight,keepaspectratio}
\IfFileExists{parskip.sty}{%
\usepackage{parskip}
}{% else
\setlength{\parindent}{0pt}
\setlength{\parskip}{6pt plus 2pt minus 1pt}
}
\setlength{\emergencystretch}{3em}  % prevent overfull lines
\providecommand{\tightlist}{%
  \setlength{\itemsep}{0pt}\setlength{\parskip}{0pt}}
\setcounter{secnumdepth}{0}
% Redefines (sub)paragraphs to behave more like sections
\ifx\paragraph\undefined\else
\let\oldparagraph\paragraph
\renewcommand{\paragraph}[1]{\oldparagraph{#1}\mbox{}}
\fi
\ifx\subparagraph\undefined\else
\let\oldsubparagraph\subparagraph
\renewcommand{\subparagraph}[1]{\oldsubparagraph{#1}\mbox{}}
\fi

%%% Use protect on footnotes to avoid problems with footnotes in titles
\let\rmarkdownfootnote\footnote%
\def\footnote{\protect\rmarkdownfootnote}

%%% Change title format to be more compact
\usepackage{titling}

% Create subtitle command for use in maketitle
\newcommand{\subtitle}[1]{
  \posttitle{
    \begin{center}\large#1\end{center}
    }
}

\setlength{\droptitle}{-2em}
  \title{An Analysis of the Premium Associated with Bitcoin}
  \pretitle{\vspace{\droptitle}\centering\huge}
  \posttitle{\par}
  \author{Jenna Christensen and Kai Jensen\\
ECON 4008-01: Macro Modeling\\
Department of Economics\\
St.~Lawrence University\\
23 Romoda Drive\\
Canton, New York 13617}
  \preauthor{\centering\large\emph}
  \postauthor{\par}
  \date{}
  \predate{}\postdate{}


\begin{document}
\maketitle
\begin{abstract}
Insert abstract here and then add JEL Codes \par
\end{abstract}

\section{Introduction}\label{introduction}

Nearly a decade ago, peer to peer payment networks and digital
currencies were unknown to virtual communities and the general
population. Cryptocurrencies including Bitcoin, LiteCoin, and DogeCoin
became household names and grasped the attention of investors, analysts,
and economists in 2016. Bitcoin's anonymous users and encrypted
transactions made it the most prominent of virtual currencies. At its
inception, Bitcoin traded at 0.63 U.S. dollars per Bitcoin, and by 2014
its value peaked at 1,101.65 U.S. dollars per Bitcoin. By November 20,
2017, its price was recorded at a record high of \$8,237.45 and
increased over 111.19\% in the 15 preceding weeks. However, its price
today is less than half of that record value. The fundamental
determinants of Bitcoin's price include supply and demand interactions,
individuals' expectations, and the development of technological
instruments used for conducting Bitcoin transactions.

Unlike standard fiat money, Bitcoin is not within the domain of central
governments, authorities, or individuals. The supply and demand for most
monetary units are driven by macroeconomic variables which include
interest rates, inflation, and the actions taken by central authorities.
However, significant changes in the price of Bitcoin are attributable to
specific factors relating to cryptocurrencies. Since Bitcoin's supply
evolves according to a publicly known algorithm and is fairly inelastic,
and the demand side of the market is mainly driven by the expectations
of investors who plan on holding the currency and later selling it,
Bitcoin has an exceptionally volatile behavior which makes it an
extremely risky yet profitable investment. Its market performance
includes steep increases and precipitous declines in value further
suggesting the market is driven by the expectations of investors and
spectators.   This empirical work seeks to analyze the shock process of
Bitcoin as well as the shock process of relatively risk-free U.S.
Treasury Bills (i.e.~to answer the following question: What is the shock
process of Bitcoin and the shock process of a relatively risk-free U.S.
Treasury Bill?). Furthermore, this study seeks to examine the risk
aversion parameter necessary to generate the level of equity return
observed in the historical price data of Bitcoin and U.S. Treasury Bills
(i.e.~to answer the following question: What is the risk aversion
parameter necessary to generate the level of equity return observed in
historical price data of Bitcoin and U.S. Treasury Bills?).
Traditionally, studies that replicate the equity premium puzzle with a
Lucas Asset Pricing Model examine the excess returns of a risky security
or index relative to those of risk-free assets or treasury bonds. Until
2016, cryptocurrencies were largely unacknowledged by academics.
Although the volatile behavior of cryptocurrency is now at the forefront
of many financial economic works, the risk premia necessary to hold
cryptocurrencies are scantly studied.

\section{Literature Review}\label{literature-review}

Robert Lucas (1978) examines the stochastic behavior of equilibrium
prices in a representative, pure exchange, single good economy with
identical consumers. His paper first examines the behavior of asset
prices in a one-good pure exchange economy with identicalconsumers and
introduces a method of constructing equilibrium prices. Lucas later
defines the general equilibrium as a pair of functions: a price function
and an optimum value function. To reach a competitive equilibrium, all
output must be consumed, all asset shares must be held, and all asset
prices must solve the dynamic program. Thus, the general equilibrium and
market clearing price for trees at time 𝑡 must satisfy the following
\emph{Insert Requirements} Hence, the equilibrium price of the asset
must satisfy \emph{insert something} Lusas' paper was the first of its
kind to model risky asset ownership decisions and determine how risk
premiums are incorporated in the price of an asset.

Subsequent to the publication of the Lucas Asset Pricing Model, Mehra
and Prescott (1985) present the equity premium puzzle. They find that in
a competitive pure exchange economy, the average annual yield of equity
is, at most, four-tenths of a percent higher than that of short-term
debt. In stark contrast, the historical yield observed by Mehra and
Prescott has a premium of six percent when accounting for U.S. business
cycle fluctuations and reasonable risk aversion levels. They conclude
that the historical U.S. equity premium, the return earned by a risky
security in excess of that earned by a relatively risk-free U.S.
Treasury Bill, is not only irrational but also inexplicable. According
to Nada (2013), the economies used in Mehra and Prescott's study have a
``stationary equilibrium for growth rate process on consumption as well
as returns''. Nada maintains that the elasticity of substitution between
consumption in time period 𝑡 and time period \(t + 1\) is sufficiently
small to yield a six percent average premium, but the magnitude of the
covariance between the marginal utility of consumption and equity
returns is not sufficiently large enough to justify the equity premium
observed. Mehra and Prescott's equity premium puzzle ignited an
extensive research effort within the fields of macroeconomics and
finance. A plethora of theoretical speculations and plausible
explanations for this anomaly have been presented, but no single
solution has been widely accepted by economists.

Traditionally, studies that replicate the equity premium puzzle with a
Lucas Asset Pricing Model examine the excess returns of a risky security
or index relative to those of risk-free assets or treasury bonds.
Although virtual currencies resemble the role of money and create an
alternative environment for conducting business, it was not until 2016
that cryptocurrencies were unacknowledged by academics. Cryptocurrencies
are commonly used as methods of payments, but it is heavily debated
whether they truly function as currencies. Since the role cryptocurrency
plays is unclear to many, how cryptocurrency is regulated by financial
institutions is controversial. Vandezande (2017) claims that it is
increasingly important to analyze the behavior of cryptocurrencies as
financial tools because there are few explanations for the current
behavior of cryptocurrencies as investment tools. He analyzes the extent
to which virtual currencies are regulated within the European Union and
ascertains that cryptocurrencies have the highest risk among all types
of virtual currencies. Although Vandezande (2017) does not include
empirical tests, he further maintains that investors are not fully
informed about the risk relating to cryptocurrency investments due to
the absence of regulatory bodies and the enforcement of protection
mechanisms5. He lastly suggests that legal frameworks used for
traditional currencies and financial investments are applicable to the
various types of virtual currencies and cryptocurrency service
providers.

Much of the financial literature contains ambiguous results concerning
the behavior of cryptocurrencies. Thus, the debate about whether
cryptocurrencies are a speculative investment asset or a currency
remains ongoing. Corbet, Meegan, et al. (2018) examine the relationships
between cryptocurrencies and other financial assets with the Diebold and
Yilmaz methodology, Barunik and Krehlik methodology, and a standard
Multivariate Generalized Autoregressive Conditional Heteroskedasticity
model with dynamic conditional correlations (MVGARCH- DCC) model. They
hypothesize that, ``cryptocurrency markets, i.e.~Bitcoin, Ripple, and
Litecoin, are strongly interconnected and demonstrate similar patterns
of return and volatility transmission with other assets.'' To study the
return and volatility transmission among Bitcoin, Ripple, and Litecoin
and research the excess return and volatility transmission to gold,
bond, equities, and the global volatility index (VIX), they measure
changes in the correlations of the aforementioned assets' volatilities
and returns. Their findings demonstrate that cryptocurrencies are
relatively isolated from market shocks and decoupled from popular
financial assets, despite the fact that the performance of each
cryptocurrency is correlated to the performance of other
cryptocurrencies. Corbet, Meegan, et al. (2018) also find that Bitcoin,
Ripple, and Litecoin are highly sensitive to industry regulations and
technological malfunctions. Ergo, the interconnectedness among
cryptocurrencies indicates that substantial changes in cryptocurrency
prices are attributable to speculative activity. These results suggest
that cryptocurrencies can be effective tools for portfolio
diversification.

Although cryptocurrencies may serve as useful portfolio diversifiers
their returns do not behave similarly to standard asset classes. Liu and
Tsyvinski (2018) investigate whether the cryptocurrency market behaves
similarly to the stock market. They do so by determining whether or not
the returns of cryptocurrency are compensated by risk factors derived
from the stock market and analyzing CAPM alphas, CAPM betas, and Eugene
Fama and Kenneth French's five risk factors. Thereafter, they study the
exposure of cryptocurrency returns to the Australian Dollar, Canadian
Dollar, Euro, Singaporean Dollar, and UK Pound. Although major national
currencies strongly comove with one another, the exposures of all
cryptocurrencies to major currencies are not statistically significant.
Hence, Liu and Tsyvinski (2018) fail to reject the null hypothesis that
cryptocurrency serves as another medium of exchange. They also examine
the exposure of cryptocurrency returns to precious metal commodities and
test whether or not cryptocurrencies serve as a store of value. Again,
they find that the exposure of cryptocurrencies is insignificant.
Traditional currencies fulfill three objectives: a unit of account, a
store of value, and a medium of exchange. However, the implementation of
empirical asset pricing models and the analysis of co-movements among
Bitcoin, Ripple, Ethereum, stocks, currencies, commodities,
macroeconomic factors, and cryptocurrency market specific factors show
that cryptocurrencies can be assessed using simple financial tools, but
they behave in a radically different manner than traditional assets. Liu
and Tsyvinski (2018) lastly conclude that only cryptocurrency market
specific factors including momentum and investor attention consistently
explain market returns.

\section{Model Environment}\label{model-environment}

\section{Financial Model}\label{financial-model}

There are two financial assets in which the representative agent can
invest: a risk-free Treasury Bill, \(b_t\), and cryptocurrency, \(s_t\).
By borrowing or saving with the asset \(b_t\), the agent subsequently
earns a fixed risk-free rate or return, \(R_f\).

The premium associated with Bitcoin is defined as \(\mu\), and the
stochastic return of Bitcoin follows a Markov Chain formally described
as \[z_i \in  Z = \{z_1, z_2, z_3, ... z_{N_z}\}\] The probability of
landing in state \(z_j \in z\) is defined as
\[\pi_{i,j} = P\{z' = z_j | z = z_i\}\] Furthermore, the persistence of
aggregate shock, \(\rho\), follows \[log(z) = \rho log(z) + \epsilon\]
where the distribution of \(\epsilon\) is defined as
\(N(0, \sigma_\epsilon^2)\). Hence, the return of Bitcoin is defined as
\[R_s = z(R_f + \mu)\].

Given financial investments \(b_{t+1}\) and \(s_{t+1}\), we compute
financial wealth during period \(t + 1\) as
\[x_{t+1} = R_jb_{t+1} + R_{s, t + 1} s_{t+1}\] The representative agent
must choose to save with risk-free Treasury Bonds or to invest in
Bitcoin. Provided that the representative agent has a utility function
following \[u(c) = \frac{c^{1-\gamma}}{1 - \gamma}\] with consumption,
\(c\), and a risk aversion parameter of \(\gamma\), a partial
equilibrium in the infinite horizon model is met when the following
value function in time \(t\) is maximized. To solve for a partial
equilibrium, we maximize the value function during time \(t\) by
choosing how much to invest in Bitcoin and Treasury Bills during period
\(t+1\).

\[V (b, s) = max_{b',s'} \{\frac{c^{1-\gamma}}{1 - \gamma} + \mathbb{E}(\beta V(b', s')) \}\ \]

subject to

\[c + b' + s' = R_jb + R_ss\] \[b' \geq b\] \[s' > 0\]

\section{Data}\label{data}

the returns of cryptocurrency, \(R_s\), are calculated at time \(t\) by
\[R_{t} = ln(\frac{P_t}{P_{t-1}})\] with \(P_t\) equal to the price of
the Bitcoin price on day \(t\). Bitcoin earns a stochastic return
\[R_{s, t+1} - R_f = \mu + \eta_{t+1}\] during period \(t + 1\) where
\(\eta_{t+1}\) is assumed to be the innovation to excess returns
distributed as \(N(0, \sigma_\eta^2)\).

\section{Results}\label{results}

\section{Discussion and Conclusion}\label{discussion-and-conclusion}

\section{References}\label{references}


\end{document}
